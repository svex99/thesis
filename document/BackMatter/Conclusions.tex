\begin{conclusions}
Se analizó el estado del arte de las soluciones DNS actuales y diferentes tecnologías de desarrollo web. A partir de ello, se identificó que los servidores DNS de código abierto, en su mayoría, no disponen de una API HTTP que facilite de forma programática su configuración. Sin embargo, todos comparten aspectos comunes de acuerdo a la definición actual de DNS, lo que ofrece una vía para generalizar sobre ellos.

Se propuso una arquitectura basada en dos funciones, \verb|load| y \verb|save|, para gestionar el sistema de almacenamiento de los servidores DNS de código abierto. Dichas funciones tienen como objetivo cargar la configuración del servidor DNS y escribir cambios al disco, respectivamente. Estas se encuentran ubicada en un entorno más amplio, orientado a microservicios, que sienta las bases para la interacción entre la API HTTP, el sistema de almacenamiento del servidor DNS, y dicho servidor.

Se escogió a BIND 9 para desarrollar la implementación y los experimentos, como muestra de la funcionalidad de la propuesta teórica, y para aplicar la solución de forma práctica sobre el entorno de La Universidad de La Habana, dónde se usa dicho software como solución DNS. Se usó Go y Gin para la implementación de la API, un proceso de \textit{parsing} para la manipulación de la configuración DNS, Docker para el despliegue de los microservicios, y Vue.js para la implementación de la aplicación web.

Las pruebas realizadas comprobaron la validez de la propuesta teórica y la efectividad de la implementación sobre BIND 9. En este sentido se discutió y aprobó la factibilidad de aplicar el software al servicio DNS de la Universidad de La Habana. La interfaz web ofrece la posibilidad de realizar modificaciones de forma más simple sobre BIND 9, por personal no especializado en este, o con acceso directo a su sistema de archivos.
\end{conclusions}
