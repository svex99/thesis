\begin{recomendations}
El sistema implementado sobre BIND 9 no es capaz de cubrir la definición completa de DNS actual. Esta es demasiado extensa para tratarla en el tiempo dispuesto por esta tesis. Por tanto, se recomienda extender los \textit{parsers} creados para los archivos de almacenamiento de BIND 9, tanto para los de configuración, como de zona. La extensión de estos, para aceptar nuevos registros de zona, aumentaría la funcionalidad de la implementación.

Se propone, además, incorporar al sistema un \textit{driver} para administrar un volumen de Docker remoto, lo que permitiría ubicar la API y el servidor de BIND en motores distintos. El \textit{driver} haría posible ubicar el volumen que almacena la configuración del servidor DNS en un servicio externo y que tanto la API como BIND puedan acceder a él. Esto facilitaría un mejor monitoreo de los recursos para el servidor de BIND 9, y mantener este servicio crítico de forma aislada ante posibles vulnerabilidades en la API o insuficientes recursos para ambos bajo un mismo \textit{host}.

Adicionalmente, se propone implementar la propuesta teórica sobre otros servidores de nombres autoritarios de software código abierto. Documentar el proceso y realizar una comparación con la implementación desarrollada sobre BIND 9. 
\end{recomendations}
