\begin{opinion}
El contenido del trabajo desarrollado por el alumno está en correspondencia con la tarea planteada. Con el trabajo el educando demostró la viabilidad de crear un sistema que permita humanizar la gestión de dominios cuando estamos en presencia de servidores de DNS de código abierto como BIND 9. El tema investigado posee actualidad ya que la gestión ágil y automatizada de los dominios dota a las entidades de una alta eficiencia. En el caso de la Universidad de la Habana, debido al gran número y diversidad de las tareas que se realizan en el entorno virtual, constantemente se realizan cambios a la configuración del dominio. Estos cambios que son críticos para la estabilidad de los servicios informáticos que se prestan en la institución son efectuados mediante la modificación directa de los archivos de configuración por parte de un administrador con experiencia. El método descrito a menudo genera errores los cuales son difíciles de detectar y corregir y requieren de la intervención de personal con experiencia que no siempre están disponibles.
Con el fin de implementar la solución, el alumno demostró poseer, no solo los conocimientos adquiridos en la carrera, sino además de un nivel adecuado de preparación a partir de haber realizado una investigación profunda de diferentes lenguajes de programación y tecnologías como Golang, Docker, BIND 9, API-Rest, Vue.js y otros.

La memoria descriptiva está bien estructurada, expone todos los elementos exigidos presentados con calidad y claridad, presenta un correcto modelado e implementación del problema planteado. Se muestra dominio de lenguajes y de las técnicas de programación. El alumno mostró muy buena capacidad para comprender los aspectos planteados por los tutores lo que incidió en la calidad durante la elaboración del software y la memoria descriptiva. Fueron cumplidos los objetivos propuestos en el trabajo y se concluyó con la etapa de implementación. El alumno cumplió con los plazos para la realización de las diferentes etapas del proyecto de diploma.
La solución propuesta por el educando constituye un elemento importante para el mejoramiento de los servicios que se brindan en el nodo central de la Universidad de la Habana, así como cualquier otra entidad que use BIND 9 como servidor DNS. El trabajo debe continuarse y agregarle funcionalidades como la solicitud automatizada de nombres de dominios y otras que puedan tener utilidad de cara a la automatización de este servicio.
Consideramos que el trabajo realizado fue satisfactorio y que el educando mostró conocimientos y habilidades que certifican su preparación como licenciado en Ciencia de la Computación. 
Por el trabajo y los resultados obtenidos y teniendo en cuenta la independencia con que trabajó el educando proponemos la calificación de EXCELENTE (5 puntos).
\end{opinion}

\vspace{1cm}

\begin{figure}[!ht]
    \centering
    \begin{subfigure}{0.16\textwidth}
        \includegraphics[width=\textwidth]{Graphics/signatures/sign1-marti.jpg}
        \caption*{\mbox{Lic. Roberto Martí Cedeño}}
    \end{subfigure}
    \hspace{4cm}
    \begin{subfigure}{0.18\textwidth}
        \includegraphics[width=\textwidth]{Graphics/signatures/masso.jpg}
        \caption*{\mbox{Lic. Alexi Massó Muñoz}}
    \end{subfigure}
\end{figure}
