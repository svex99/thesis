\chapter{Detalles de Implementación y Experimentos}\label{chapter:implementation}

Con el objetivo de llevar a cabo la experimentación en la implementación de una API para los servidores DNS de código abierto, se tomó como software representativo BIND 9. En este capítulo se explican sus características principales que interactúan en la propuesta y la implementación de las funciones \verb|load| y \verb|save| sobre él.

En el presente capítulo se discutirán primeramente las características del sistema de almacenamiento que usa BIND para almacenar la configuración y las zonas del servidor DNS. Luego, en la sección 4.2, el mecanismo mediante el cual se notifica a BIND que ha sido modificado sus sistema de configuración, para que así pueda reflejar los cambios.

En las secciones posteriores, de la 4.3 a la 4.5, se describe como fueron desarrolladas la funciones de la propuesta y qué alternativas fue necesario tomar en su implementación. También se muestran las principales consideraciones para manejar errores ante los pedidos y cómo garantizar la sincronización entre los datos que maneja la API y el servidor de nombres.

En la sección 4.6, se discute el despliegue de los servicios con Docker y Compose, así como las imágenes usadas en el proceso. Finalmente, se plantean los experimentos realizados y se discuten sus resultados.

\section{Sistema de Almacenamiento}\label{sec:bind-storage}
BIND usa como sistema de almacenamiento para el servidor DNS archivos de texto plano. En estos archivos se almacena tanto la configuración del servidor autoritario o secundario, como los archivos de zona con sus registros. En la imagen de Docker de BIND basada en Ubuntu\footnote{\url{https://hub.docker.com/r/ubuntu/bind9}} se pueden encontrar los archivos de configuración y de zona en los directorios \verb+/etc/bind+ y \verb+/var/lib/bind+ respectivamente.

\begin{lstlisting}[frame=single, numbers=none, caption=Contenido del directorio \textbf{/etc/bind}.]
$ ls /etc/bind
bind.keys  db.0  db.127  db.255  db.empty  db.local  named.conf
named.conf.default-zones  named.conf.local  named.conf.options
rndc.key  zones.rfc1918
\end{lstlisting}

El fichero principal de configuración para BIND es \verb+named.conf+, desde él se importan con la directiva \verb+include+ los demás ficheros de configuración (\verb|named.conf.options|, \verb|named.conf.local| y \verb|named.conf.default-zones|) que se encuentran en el directorio. Así, la configuración puede ser dividida en diferentes archivos, facilitando su mantenibilidad y legibilidad.

El archivo \verb+named.conf.options+ tiene las opciones globales que son usadas al iniciar el servicio. Estas son ubicadas entre llaves en el cuerpo de la declaración \verb+options+. La gramática está disponible online en el manual de administrador para BIND\footnote{\url{https://bind9.readthedocs.io/en/v9_18_8/reference.html\#namedconf-statement-options}}. En este se manejan los aspectos fundamentales de DNS, \textit{caching}, logging a través de \verb+dnstap+, así como la configuración relacionada a DNSSEC. En este espacio también son definidas las listas de control de acceso (ACL, por sus siglas en inglés), que permiten restringir el acceso al servidor basado en la dirección IP del host que hace la petición. BIND dispone de gran granularidad al poder definir un ACL para cada uno de diferentes tipos de consultas que acepta el servidor.

El fichero \verb+named.conf.local+ contiene la configuración local para el servidor DNS y es donde se declaran las zonas que este va a manejar. En este también se puede hacer uso de la directiva \verb+include+, para importar otros ficheros que declaran zonas usualmente, como es el caso de \verb+zones.rfc1918+.

Una zona es definida con un nombre, una clase (por defecto \verb+IN+, por \verb+Internet+) y un cuerpo que contiene un conjunto de opciones para la zona. Dentro de estas opciones es requerido definir su tipo con la palabra clave \verb+type+. Cada tipo tiene una gramática distinta de acuerdo a las opciones que admite.

La zona dentro de sus opciones permiten definir diferentes ACL, similar a las opciones globales. Cada zona puede tener su política para manejar actualizaciones en tiempo de ejecución. Específicamente, la opción \verb+allow-update+ maneja mediante una ACL quien pude actualizar registros en una zona. De igual forma, \verb+update-policy+ permite controlar quien modifica la zona basada en la identidad del solicitante, la identidad es determinada por la clave que firmó el pedido, usando TSIG o SIG(0). Estos dos mecanismos para controlar el acceso se definen de forma excluyente y el segundo solo puede ser usado en zonas de tipo primario (\verb+primary+ o \verb+master+).

Los otros ficheros ubicados en \verb+/etc/bind+, son \verb+bind.keys+ y el grupo de \verb+db.*+. El primero tiene como objetivo sobreescribir las claves públicas (\textit{trust anchors}) que usa BIND para DNSSEC. Los segundos son empleados para indicar la zona del \textit{host} y la interfaces de \textit{loopback} y \textit{broadcast}. Estos son cargados directamente por \verb+named.conf.default-zones+.

\begin{lstlisting}[frame=single, numbers=none, caption=Contenido del fichero \textbf{named.conf.default-zones}]
$ cat /etc/bind/named.conf.default-zones
// prime the server with knowledge of the root servers
zone "." {
    type hint;
    file "/usr/share/dns/root.hints";
};

// be authoritative for the localhost forward and reverse zones,
// and for broadcast zones as per RFC 1912

zone "localhost" {
    type master;
    file "/etc/bind/db.local";
};

zone "127.in-addr.arpa" {
    type master;
    file "/etc/bind/db.127";
};

zone "0.in-addr.arpa" {
    type master;
    file "/etc/bind/db.0";
};

zone "255.in-addr.arpa" {
    type master;
    file "/etc/bind/db.255";
};
\end{lstlisting}

\section{Comunicación con el servidor DNS}\label{section:comm-bind}

Cuando BIND sufre cambios en los ficheros de configuración estos no son detectados y reflejados en tiempo de ejecución. Debe reiniciarse el servicio o notificarlo de que ha ocurrido una actualización. La alternativa más eficiente es la segunda y \verb+rndc+ es la herramienta recomendada para este proceso.

\verb+rndc+ se comunica con el servidor de nombres a través de una conexión TCP, enviando comandos con una firma digital. La versión actual utiliza como algoritmos de autenticación HMAC-MD5 (por compatibilidad), HMAC-SHA1, HMAC-SHA224, HMAC-SHA256 (por defecto), HMAC-SHA384, y HMAC-SHA512. Estos utilizan una clave secreta compartida en cada lado de la conexión, lo cual provee de autenticación de tipo TSIG para la petición del comando y la respuesta del servidor. \verb+rndc+ utiliza una archivo de configuración para determinar cómo conectarse al servidor y el algoritmo y llave a emplear en la autenticación.

Para indicar cambios en los archivos de BIND con \verb|rndc| están disponibles los comandos \verb|reconfig| y \verb|reload|. El primero es empleado para notificar de una actualización en la configuración (\verb|named.conf| o sus dependencias) de BIND, por tanto, si una nueva zona es añadida BIND lo detectará. A pesar de cargar las nuevas zonas, \verb|reconfig| no percibirá cambios en las zonas. Para detectar cambios en la zona es necesario usar \verb|reload| o, de forma más específica, \verb|reload <zone-name>|. Especificando la zona se agiliza el proceso en servidores que tengan un alto volumen de zonas, evitando verificar zonas innecesariamente.

\begin{table}[!ht]
    \centering
    \begin{tabular}{|l|l|}
    \hline
        \textbf{Evento} & \textbf{Comando} \\ \hline
        Nueva zona & \verb|reconfig| \\ \hline
        Modificar zona & \verb|reload <zone-name>| \\ \hline
        Eliminar zona & \verb|reconfig| \\ \hline
    \end{tabular}
    \caption{Comando \textbf{rdnc} apropiado para cada evento.}
    \label{table:rndc-event}
\end{table}

En las opciones de conexión de \verb|rndc| puede ser especificada la dirección IP del servidor de BIND al que conectarse. Siendo así, es posible instalar \verb|rdnc| en el ambiente de la API y especificar la dirección de BIND en el contenedor. Pero dado que la instalación de BIND viene con \verb|rndc| por defecto, es más factible invocar este de forma programática a través de Docker. Con este fin, se implementó un método que ejecutase un comando en el contenedor remoto, tomando como referencia la implementación del comando \verb|docker exec|\footnote{\url{https://github.com/docker/cli/blob/1163b4609978e0e6f2b2629b59c4a62d348e1466/cli/command/container/exec.go\#L99}} y en base a él los métodos \verb|Reconfig| y \verb|ReloadZone|.

\begin{lstlisting}[frame=single, language=Go, caption=Métodos para actualizar a BIND con los cambios en sus archivos.]
func (bs *BindService) exec(command ...string) error {
    if _, err := bs.DockerCli.ContainerInspect(bs.ctx, bs.ContainerId); err != nil {
        return err
    }

    execCreateConfig := &types.ExecConfig{
        User:         "bind",
        Privileged:   false,
        Tty:          false,
        AttachStdin:  false,
		AttachStderr: true,
        AttachStdout: false,
		Detach:       false,
        DetachKeys:   "",
        Env:          []string{},
        WorkingDir:   "/",
        Cmd:          command,
    }

    response, err := bs.DockerCli.ContainerExecCreate(bs.ctx, bs.ContainerId, *execCreateConfig)
    if err != nil {
        return err
    }
    if response.ID == "" {
        return errors.New("exec ID empty")
    }

    execStartConfig := &types.ExecStartCheck{
        Detach: execCreateConfig.Detach,
        Tty:    execCreateConfig.Tty,
    }

    if err := bs.DockerCli.ContainerExecStart(bs.ctx, response.ID, *execStartConfig); err != nil {
        return err
    }

    return nil
}

// Runs `rndc reconfig` in the BIND server.
// Reloads the configuration file and loads new zones, but does not reload existing zone files even if they have changed.
func (bs *BindService) Reconfig() error {
	return bs.exec("rndc", "reconfig")
}

// Runs `rndc reload {zone}` in the BIND server
func (bs *BindService) ReloadZone(zone string) error {
	return bs.exec("rndc", "reload", zone)
}
\end{lstlisting}

\section{Lectura de la Configuración DNS}

Con el fin de implementar la función \verb+load+ sobre el sistema de almacenamiento de BIND se hace necesario \textit{parsear} estos archivos de texto plano. Se requiere de un \textit{parser} para los archivos que contienen registros y otro para los archivos de configuración, específicamente \verb+named.conf.local+, que es el que usualmente contiene las zonas añadidas por los administradores del servidor.

El proceso de \textit{parsing} permite estructurar la información en texto plano tanto al leerla como al escribirla a disco. La gramática para los archivos de configuración está completamente definida en la documentación de BIND y está sujeta a los cambios que introduzca ISC en el software. En el caso de los archivos de registros no es tan sencillo, se rigen por DNS y este es un sistema en constante cambio, son añadidos nuevos tipos de registros o especificaciones, por lo que la gramática puede cambiar en un futuro. Definirla sobre los tipos más comunes y dejar como alternativa su extensibilidad es la opción más factible en este proyecto.

Para la implementación de los \textit{parsers} fue empleada la biblioteca de Go, Participle\footnote{\url{https://github.com/alecthomas/participle}}. Esta hace uso de la genericidad en el lenguaje para la definición de los \textit{parsers} y genera un \textit{parser} de descenso recursivo con \textit{backtracking} a partir de la gramática de entrada. La gramática es definida de forma declarativa en las etiquetas (\verb|parser|) de las estructuras que la conforman y lo tokens son almacenados en el campo al que referencian.

El \textit{parser} resultante para los archivos de zona, de forma simplificada, es el siguiente:

\begin{lstlisting}[frame=single, language=Go, escapechar=!, caption=Implementación en Go del \textit{parser} para los archivos de zona.]
import (
    "github.com/alecthomas/participle/v2"
	"github.com/alecthomas/participle/v2/lexer"
)

var (
    ZoneLexer = lexer.MustSimple([]lexer.SimpleRule{
        {Name: "Directive", Pattern: `\$(ORIGIN|TTL)`},
        {Name: "Keyword", Pattern: `@|IN`},
        {Name: "RType", Pattern: `SOA|NS|A|MX|TXT|CNAME`},
        {Name: "Origin", Pattern: `[a-zA-Z][\w\-]*\.[a-zA-Z]+`},
        {Name: "Name", Pattern: `[a-zA-Z][\w\-]*`},
        {Name: "Ttl", Pattern: `\d+[hdw]`},
        {Name: "Ipv4", Pattern: `\d{1,3}\.\d{1,3}\.\d{1,3}\.\d{1,3}`},
        {Name: "Uint", Pattern: `\d+`},
        {Name: "String", Pattern: `"[^"\n]*"`},
        {Name: "Punct", Pattern: `[\.\(\)]`},
        {Name: "Comment", Pattern: `;[^\n]*\n+`},
        {Name: "Whitespace", Pattern: `[ \t\r]+`},
        {Name: "NewLine", Pattern: `[\n]+`},
    })
    ZoneParser = participle.MustBuild[DomainConf](!\label{line:parser}!
        participle.Lexer(ZoneLexer),
        participle.Union[Record](NSRecord{}, ARecord{}, MXRecord{}, TXTRecord{}, CNAMERecord{}),!\label{line:union-gen}!
        participle.Elide("Whitespace", "Comment"),
        participle.Unquote("String"),
        participle.UseLookahead(2),
    )
)

type ZoneConf struct {
	Origin    string     `parser:"'$ORIGIN' @Origin '.' NewLine" json:"origin"`
	Ttl       string     `parser:"'$TTL' @Ttl NewLine" json:"ttl"`
	SOARecord *SOARecord `parser:"@@ NewLine" json:"soaRecord"`
	Records   []Record   `parser:"@@*" json:"records"`
}

type Record interface {
    // Omitted for brevity
}

type SOARecord struct {
	NameServer string `parser:"'@' 'IN' 'SOA' @Name" json:"nameServer"`
	Admin      string `parser:"@Name" json:"admin"`
	Serial     uint   `parser:"'(' @Uint" json:"serial"`
	Refresh    uint   `parser:"@Uint" json:"refresh"`
	Retry      uint   `parser:"@Uint" json:"retry"`
	Expire     uint   `parser:"@Uint" json:"expire"`
	Minimum    uint   `parser:"@Uint ')'" json:"minimum"`
}

type NSRecord struct {!\label{line:ns}!
	Type       string `parser:"'@' 'IN' @'NS'" json:"type"`
	NameServer string `parser:"@Name NewLine" json:"nameServer"`
}

type ARecord struct {
	Name string `parser:"@Name" json:"name"`
	Type string `parser:"'IN' @'A'" json:"type"`
	Ip   string `parser:"@Ipv4 NewLine" json:"ip"`
}

type MXRecord struct {
	Type        string `parser:"'@' 'IN' @'MX'" json:"type"`
	Priority    uint   `parser:"@Uint" json:"priority"`
	EmailServer string `parser:"@Name NewLine" json:"emailServer"`
}

type TXTRecord struct {
	Type  string `parser:"'@' 'IN' @'TXT'" json:"type"`
	Value string `parser:"@String NewLine" json:"value"`
}

type CNAMERecord struct {
	SrcName string `parser:"@Name 'IN'" json:"srcName"`
	Type    string `parser:"@'CNAME'" json:"type"`
	DstName string `parser:"@Name NewLine" json:"dstName"`
}
\end{lstlisting}

Como los diferentes tipos de registros (excluyendo a \verb|SOA|, por conveniencia siempre el primero) pueden encontrarse indistintamente en un archivo de zona, el orden de estos no es relevante, solo la información contenida acorde al tipo. En la línea~\ref{line:union-gen} se captura este comportamiento haciendo uso de la genericidad que ofrece el lenguaje. Por esta vía es posible declarar una producción en la gramática de la forma \verb+Record -> NSRecord | ARecord | MXRecord | TXTRecord | CNAMERecord+. El nodo raíz del AST (\textit{Abstract Syntax Tree}) es la estructura usada como tipo genérico en la construcción del \textit{parser} (línea~\ref{line:parser}), y en este caso particular, es la base para la información del archivo de zona. Aquí es declarado que todo archivos de zona contiene \verb+$TTL+, \verb|$ORIGIN|, \verb|SOA| y un \textit{slice} de \verb|Record|.

Tomando como ejemplo la estructura \verb+NSRecord+ (línea~\ref{line:ns}), esta va a consumir una cadena con una forma similar a \verb|@ IN NS <Name>|. Como resultado, va a almacenar en \verb|Type| el valor de tipo \verb|string| \verb|"NS"|. Luego declara que en el campo \verb+NameServer+ se va almacenar un token que coincida con el tipo \verb+Name+ del \textit{lexer} y consumir un salto de línea.

Una ventaja notable de poder representar la gramática en las etiquetas de las estructuras es que es consistente con la sintaxis usada para la serialización por defecto de estructuras en Go. Así se puede mantener en cada uno de los atributos el proceso de \textit{parsing} relacionado al \textit{token} que va almacenar, y junto ello, la etiquetas usadas para serializar y deserializar las estructuras al ser enviadas por la red.

% Así se puede mantener en la misma estructura que almacena los tokens resultado del proceso de \textit{parsing} la declaración de los campos correspondientes cuando se serialicen y deserialicen las estructuras al ser transmitidas por la red.

La implementación de la función \verb|load| se puede dividir en dos partes. La primera, \textit{parsear} la configuración (\verb|/etc/bind/named.conf.local|) de BIND para detectar todas las zonas disponibles y una segunda, en la que para cada zona encontrada se \textit{parsea} el archivo (campo \verb|file|) al que esta refiere.

\begin{lstlisting}[frame=single, language=Go, caption=Implementación de la función \textbf{load} para BIND.]
func (bs *BindService) Load() {
	var err error

	bs.BindConf, err = bs.parseBindConf(Service.ZonesFilePath)
	if err != nil {
		log.Fatal(err)
	}

	bs.Domains = make(map[string]*parser.DomainConf)

	for _, zone := range Service.BindConf.Zones {
		filename := zone.File[strings.LastIndex(zone.File, "/")+1:]

		zConf, err := Service.parseDomainConf(setting.Bind.LibPath + filename)
		if err != nil {
			log.Printf("Error loading %s: %s\n", filename, err)
			continue
		}

		Service.Domains[zConf.Origin] = zConf
	}
}
\end{lstlisting}

Este método es invocado al inicio del sistema y en caso de error al cargar la configuración de BIND se detiene la ejecución y se reporta. Si una zona contiene errores sintácticos es omitida a la hora de cargarla y el error es reportado. Solo son servidas zonas listadas en la configuración de BIND y que no contengan errores de sintaxis.

Las estructuras de datos resultantes almacenadas en \verb|BindService.BindConf| y \verb|BindService.Zones|, contienen la información que refleja el estado actual del servidor de nombres. El primero es el AST resultante de \textit{parsear} la configuración y el segundo, un mapa del nombre de cada zona al respectivo AST de su archivo que contiene el conjunto de RR.

\subsection{Almacenamiento en Memoria vs Disco}

Para la solución, inicialmente fue considerada cargar la configuración del servidor DNS a una base de datos. Dicha solución fue desechada en favor de almacenarla en memoria. La principal desventaja de esto es tener que implementar algunos mecanismos de validación que pueden ser declarados de forma relativamente sencilla con SQL. Pero trae varias ventajas consigo, la primera, y más importante, es no tener dos fuentes de verdad (base de datos de la API y base de datos del servidor DNS). De esta forma cada vez que se inicie la API se garantiza que la información que sirve está sincronizada con la configuración del servidor DNS. En última instancia el servidor de nombres es quien debe tener la información precisa pues es quien provee el servicio DNS. Además, se pude obtener menores valores de latencia, dado que el acceso a valores aleatorios en memoria es mucho más rápido que en disco (HDD o SSD)[\cite{jacobs2009pathologies}].

\section{Escritura de la Configuración DNS}

La información resultante del proceso de \textit{parsing} puede ser modificada y escrita a disco sin inconvenientes, permitiendo actualizar los archivos de configuración del servidor DNS de forma directa sin afectar su funcionamiento. Este proceso puede resumirse en tomar el AST resultante, llevarlo a una cadena de texto, similar a la que tuvo como origen, y escribir dicha cadena al fichero de configuración.

Como indica la función \verb+save+, es necesario mantener actualizada la copia de la configuración en memoria, esto evita tener que \textit{parsear} el archivo de configuración en modificaciones posteriores. Así se realiza el \textit{parsing} una sola vez al iniciar el sistema, posteriormente solo las escrituras del AST al archivo correspondiente.

En la propuesta, \verb|save| recibe como parámetros un servidor de nombres $A$ y una actualización a aplicar $u$. Dicha actualización puede ser uno de diferentes eventos, ya definidos en la Tabla \ref{table:rndc-event}, donde la modificación de una zona puede ser de forma específica añadir, modificar o eliminar uno de los registros de la zona.

Por tanto $u$, puede verse como un elemento de un conjunto de posibles eventos, y sería necesario verificar antes de procesar $u$ dentro de \verb|save| qué tipo de evento es el recibido. Una implementación que cubra esta lógica puede comprometer la legibilidad de \verb|save|, dado que genera código espagueti, además de resultar en una función más difícil de mantener y comprobar. Además, se debe tener en cuenta que no todas las modificaciones a la configuración requieren modificar la misma cantidad de archivos o invocar el mismo comando de \verb|rndc|. En el caso de añadir una nueva zona, es necesario modificar la configuración del servidor, así como el archivo de la nueva zona, mientras que al modificar la zona, solo se modifica el archivo de zona, y, como se analizó en \ref{section:comm-bind}, los comandos de \verb|rndc| a usar en cada caso son distintos.

Como consecuencia de esto, en la implementación se decide dividir \verb|save| en un conjunto de funciones (ver Figura \ref{fig:split-save}), que siguen cada uno de los estados que esta define, pero que manejan cada tipo de evento por separado. Adicionalmente, cada una de estas funciones está directamente relacionada con un punto final de la API.

\begin{figure}[!ht]
    \centering
    \includegraphics[width=\linewidth]{draws/split-save.png}
    \caption{Funciones que siguen el diseño de \textbf{save} para dividir la lógica.}
    \label{fig:split-save}
\end{figure}

Un cambio en la configuración por una función derivada de \verb|save| es originado por una invocación a los puntos finales de la API. Por tanto, cabe la posibilidad de que se traten de realizar de forma concurrente en memoria. Siendo así, se hace necesario limitar el acceso concurrente de escritura para evitar la corrupción de los datos haciendo uso de una instancia de \verb|sync.Mutex|. Cuando una modificación es recibida por la API, la función a ser invocada toma el control sobre este bloqueo, realiza la escritura en disco y memoria y luego lo libera.

La función principal del bloqueo es mantener la consistencia entre la información que se encuentra en memoria y en disco, asegurando que cada escritura en memoria tenga su sucesiva escritura en disco. El AST puede implementar su propio mecanismo para manejar el acceso concurrente en memoria, de igual forma se puede hacer a la hora de implementar la escritura en los archivos de configuración del disco. Sin embargo, estos dos por separado no garantizan la sincronización memoria-disco ante el acceso concurrente a la API. Además, es importante poder recuperarse de un error de escritura en disco, sin corromper la información en memoria.


Las zonas en los servidores DNS primarios son consideradas zonas de inicio de autoridad. Estas son la fuente de información para el servidor maestro y sus esclavos. Son representadas por un registro de tipo \verb|SOA| (\textit{Start Of Authority}) que contiene información administrativa y de importancia para los servidores secundarios. Los diferentes campos que contiene indican cada qué tiempo los servidores secundarios deben consultar al primario por actualizaciones, cada qué tiempo consultar por cambios en el número de serie y después de qué intervalo detenerse los esclavos en caso de fallo en el maestro. Un aumento en el número de serie indica a los servidores secundarios que ha ocurrido una actualización en la zona y por tanto se debe iniciar una transferencia.

\begin{lstlisting}[frame=single, numbers=none, caption=Ejemplo de zona con \textit{SOA}.]
$ORIGIN example.com.
$TTL 86400
@   IN  SOA     ns.example.com. admin@example.com (
        2022110300  ;Serial
        7200        ;Refresh
        3600        ;Retry
        1209600     ;Expire
        3600        ;Negative response caching TTL
)
\end{lstlisting}

Por tanto, es de vital importancia para el funcionamiento del servicio DNS que el número de serie sea incrementado en cada modificación a la zona, a pesar de que se pueda enviar una consulta de tipo \verb|NOTIFY| a los esclavos para alertar de una transferencia de zona. El número de serie es un entero positivo que indica la versión de la copia original de la zona, por lo general se puede usar como valor la fecha de cambio más dos dígitos para el caso en el que ocurran múltiples modificaciones el mismo día.

De esta forma, en la implementación es realizada la actualización de la versión de la zona cada vez que se persisten cambios en la configuración DNS.

\begin{lstlisting}[frame=single, language=Go]
// Generates a new serial for the SOA record.
// Generated serials follows the format YYYYMMDDNN where NN is a two digits identifier.
func (zc *ZoneConf) UpdateSerial() {
    now := time.Now().UTC()
    newSerial := uint(now.Year()*1_000_000 + int(now.Month())*10_000 + now.Day()*100)
    if zc.SOARecord.Serial >= newSerial {
        zc.SOARecord.Serial = zc.SOARecord.Serial + 1
    } else {
        zc.SOARecord.Serial = newSerial
    }
}
\end{lstlisting}

Según los aspectos definidos en esta sección, se muestra a continuación la función \verb|CreateZone|, una de las mencionadas en la Figura \ref{fig:split-save} y que refleja el comportamiento de \verb|save| y, de forma general, el conjunto de funciones en que está dividida.

\label{func:create-zone}
\begin{lstlisting}[frame=single, language=Go, escapechar=|, caption=Ejemplo de la función que añade una nueva zona al servidor DNS.]
func (bs *BindService) CreateZone(data *schemas.DomainData) (*parser.ZoneConf, error) {
    // Get write access to the filesystem and release it when done
    bs.Mutex.Lock()
    defer bs.Mutex.Unlock()

    // Validate that the new zone is not defined already 
    if _, ok := bs.Zones[data.Origin]; ok {
        return nil, fmt.Errorf("zone %s exists already", data.Origin)
    }

    // Create the new zone from received data
    zConf := &parser.ZoneConf{
        Origin: data.Origin,
        Ttl:    data.Ttl,
        SOARecord: &parser.SOARecord{
            NameServer: data.NameServer,
            Admin:      data.Admin,
            Refresh:    data.Refresh,
            Retry:      data.Retry,
            Expire:     data.Expire,
            Minimum:    data.Minimum,
        },
        Records: []parser.Record{},
    }

	bindConf := *bs.BindConf

	if err := bindConf.AddZone(zConf); err != nil {
		return nil, err
	}

    // Write new changes to BIND files and rollback on error
	rollbackZConf, err := zConf.WriteToDisk(zConf.GetFilename())
	if err != nil {
		rollbackZConf()
		return nil, err
	}

	rollbackBindConf, err := bindConf.WriteToDisk(bs.ZonesFilePath)
	if err != nil {
		rollbackZConf()
		rollbackBindConf()
		return nil, err
	}

    // Notify BIND about the new update|\label{line:call-reconf}|
	if err := bs.Reconfig(); err != nil {
		rollbackZConf()|\label{line:reconf-roll1}|
		rollbackBindConf()|\label{line:reconf-roll2}|
		return nil, err
	}

    // Sync changes on memory|\label{line:sync-mem}|
	bs.BindConf = &bindConf
	bs.Zones[zConf.Origin] = zConf

	return zConf, nil
}
\end{lstlisting}

Retomando el funcionamiento de \hyperref[proc:save]{\textbf{save}}, el método anterior primeramente toma control sobre el sistema de archivos del sistema de almacenamiento, evitando la escritura concurrente por otras invocaciones a la API. Luego verifica que la información a ser introducida en la configuración del servidor de nombres es válida, usando la que se tiene de forma local en memoria. Una vez confirmado que los cambios pueden ser persistidos, pasa a efectuarse la escritura en disco, en este caso en dos archivos (\verb|/etc/bin/named.conf.local| y \verb|/var/cache/lib/db.<zone-name>|). Cuando es efectuada la escritura de la configuración, se debe notificar a BIND de los cambios, en este caso haciendo uso del comando \verb|rndc reconfig| a través de Docker, invocación realizada por \verb|Reconfig| (línea \ref{line:call-reconf}). El último estado de \verb|save| es ejecutado a partir de la línea \ref{line:sync-mem}, donde se almacenan en memoria las estructuras que validaron y aplicaron los cambios a la configuración antes de ser persistida.

\section{Manejo de errores}

Ambos \textit{parsers} hacen uso de su respectivo método \verb|WriteToDisk| para llevar el AST resultante a una cadena de texto y escribirla en un archivo de texto plano. Esta función puede lanzar una excepción si algún proceso externo a la API tiene bloqueado el recurso a la hora de escribir. Por tanto, por conveniencia \verb|WriteToDisk| retorna una función que puede ser invocada para ``recuperar'' el sistema de archivos en caso de error y que sea interrumpido el proceso de escritura.

\begin{lstlisting}[frame=single, language=Go, caption=Implementación de \textbf{WriteToDisk} para el AST de una zona.]
import (
    "fmt"
    "os"
    "strings"

    "github.com/svex99/bind-api/pkg/file"
)

// Writes the zone configuration to a plain text file.
// Returns a function that rollbacks the process.
func (zc *ZoneConf) WriteToDisk(filename string) (func(), error) {
	zc.UpdateSerial()

	content := []string{
		fmt.Sprintf("$ORIGIN %s.", zc.Origin),
		fmt.Sprintf("$TTL %s", zc.Ttl),
		fmt.Sprintf(
			"@ IN SOA %s %s ( %d %d %d %d %d )\n",
			zc.SOARecord.NameServer, zc.SOARecord.Admin,
			zc.SOARecord.Serial, zc.SOARecord.Refresh, zc.SOARecord.Retry, zc.SOARecord.Expire, zc.SOARecord.Minimum,
		),
	}
	for _, record := range zc.Records {
		content = append(content, record.String())
	}

    // Create a backup of config if file exists
    rollback := file.MakeBackup(filename)

    if err := os.WriteFile(filename, []byte(strings.Join(content, "\n")), 0666); err != nil {
        return rollback, err
    }

    return rollback, nil
}
\end{lstlisting}

La función usada para recuperar la copia de seguridad es almacenada en la variable \verb|rollback| y es retornada por la función para uso del contexto en que sea invocada. En la implementación del método \verb|CreateZone| (ver Código \ref{func:create-zone}), se puede comprender cómo esta función es invocada ante un error en procedimientos posteriores, como es la sucesiva escritura en disco o notificar al servidor de BIND sobre los cambios en la configuración.

La función \verb|MakeBackup| es la encargada de realizar la copia de seguridad de los archivos a modificar y capturar el contexto para realizar la recuperación. Dicha función se muestra a continuación.

\begin{lstlisting}[frame=single, language=Go, caption=Función encargada de mantener una copia de seguridad al modificar los archivos de BIND.]
import (
    "log"
    "os"
)

// Makes a backup (.bak file) of a plain text file
// Returns a function that rollbacks the backup file
func MakeBackup(filename string) func() {
    backup := filename + ".bak"
    bak_err := os.Rename(filename, backup)
    rollback := func() {
        if bak_err == nil {
            if err := os.Rename(backup, filename); err != nil {
                panic(err)
            }
        }
    }
    return rollback
}
\end{lstlisting}

Cuando se notifica al servidor de nombres para que actualice su servicio con los nuevos cambios es porque ya fueron modificados sus archivos de configuración. Estos cambios pueden haber sido realizados a solo un archivo de zona o adicionalmente al de configuración (ver Figura \ref{fig:split-save}). En caso de un error porque no haya comunicación con el servidor DNS, o este no este en ejecución, la copia de seguridad es recuperada y la API retorna error ante el pedido. Este comportamiento también puede ser apreciado en el ejemplo de código de \verb|CreateZone| (línea \ref{line:call-reconf}), y como ante un error se recupera tanto el archivo de zona, como el de configuración (líneas \ref{line:reconf-roll1} y \ref{line:reconf-roll2} respectivamente).

\section{Despliegue de los Servicios}

Los diferentes servicios que conforman el sistema están diseñados para ser ejecutados sobre Docker. Las diferentes ventaja de esta tecnología fueron mostradas en la sección \ref{sec:docker} del Estado del Arte. Su aplicación en este escenario facilita la portabilidad y el acceso seguro de los archivos de configuración que van a compartir la API y el servidor de BIND.

\subsection{Ejecución con Compose}

Compose como herramienta para el despliegue de sistemas con múltiples contenedores es de gran utilidad en el proceso. Desde la fase de desarrollo fue empleado para crear el ambiente donde implementar toda la funcionalidad del proyecto. Esto hace que la transición al despliegue del mismo no sea tan compleja.

Para hacer uso de Compose se declaró en un archivo \verb|compose.yaml| los diferentes servicios, volúmenes y la red privada en la que va a estar desplegada el sistema. Dicho archivo es mostrado a continuación para luego discutir sus aspectos principales.

\begin{lstlisting}[frame=single, escapechar=|, caption=\textbf{compose.yaml} creado para el manejo de los servicios en la arquitectura.]
version: "3.8"

x-default-api: &default-api
    depends_on:
        - bind-server
    dns:
        - 172.30.10.1
    environment:
        - DOCKER_HOST=http://host.docker.internal:2375
        - BIND_HOST=bind-server
    extra_hosts:
        - "host.docker.internal:host-gateway"|\label{line:extra_hosts}|
    networks:
        bind-services:
            ipv4_address: 172.30.10.2

x-default-api-dev: &default-api-dev
    # omitted for brevity

services:
    bind-api-prod:
        <<: *default-api
        image: bind-api-prod
        container_name: bind-api-prod
        build:
            dockerfile: docks/api.Dockerfile
            context: .
            target: prod
        ports:
            - "2020:2020"
        volumes:
            - bind-conf:/data/bind/conf
            - bind-lib:/data/bind/lib
            - bind-api-data:/data/api
        profiles:
            - prod

    bind-api-dev:
        <<: *default-api-dev
        # omitted for brevity

    bind-api-bench:
        <<: *default-api-dev
        # omitted for brevity

    bind-server:
        image: docker.uclv.cu/ubuntu/bind9
        container_name: bind-server
        environment:
            - TZ=UTC
            - BIND9_USER=bind
        ports:
            - "30053:53/udp"
            - "30053:53/tcp"
        volumes:
            - bind-conf:/etc/bind
            - bind-lib:/var/lib/bind
            - bind-cache:/var/cache/bind
            - bind-log:/var/log
        networks:
            bind-services:
            ipv4_address: 172.30.10.1

volumes:
    bind-conf:
        external: true
    bind-lib:
        external: true
    bind-cache: {}
    bind-log: {}
    bind-api-data:
        external: true

networks:
    bind-services:
        name: bind-services
        driver: bridge
        ipam:
            driver: default
            config:
                - subnet: 172.30.10.0/16
\end{lstlisting}

La API comparte con BIND los volúmenes \verb|bind-conf| y \verb|bind-lib|, el primero almacena la configuración del servidor DNS, mientras que el segundo es el que va a almacenar las zonas que esté manejando el servidor de nombres. Estos volúmenes deben ser creados de forma externa con el comando \verb|docker volume create <vol-name>|. El objetivo de compartir estos volúmenes entre ambos servicios es ganar acceso al sistema de archivos de BIND 9 para alterar su sistema de almacenamiento.

Ubicar a ambos servicios en una misma red privada permite mantenerlos aislados en su comunicación del \textit{host} y garantizar su comunicación de forma más directa. Cada servicio expone solo los puertos necesarios al exterior, la API un puerto por el que va a recibir los pedidos HTTP, y BIND expone los puertos necesarios para el funcionamiento del servicio DNS, tanto TCP como UDP.

Para realizar la ejecución de los comandos de \verb|rndc| a través de la API de Docker se hace necesario añadir en el campo \verb|extra_hosts| (línea \ref{line:extra_hosts}) un mapeo de direcciones IP para poder acceder al motor de Docker de la computadora en que está siendo ejecutado el servicio de BIND 9 y la misma aplicación (Figura \ref{fig:extra_hosts}). Si el contenedor de BIND 9 está ubicado en  un motor de Docker remoto, la dirección IP del \textit{host} debe ser especificada en lugar de \verb|host-gateway|.

\begin{figure}[!ht]
    \centering
    \includegraphics[width=\linewidth]{draws/extra_hosts.png}
    \caption{Ejecución de comandos desde la API al servidor de BIND 9 a través del motor de Docker que gestiona a ambos.}
    \label{fig:extra_hosts}
\end{figure}

Para el servicio de BIND 9 fue usada la imagen que ofrece Ubuntu en DockerHub, ya mencionada en la sección \ref{sec:bind-storage}. La imagen de la API fue creada usando construcción de varias etapas (\textit{multi-stage build}) para minimizar considerablemente el tamaño de la imagen en producción [\cite{multi-stage}].

\begin{lstlisting}[frame=single, escapechar=|, caption=Dockerfile para la API.]
FROM docker.uclv.cu/golang:1.19.2 as base

ENV CGO_ENABLED=0

WORKDIR /go/src

COPY . .

RUN go build -mod=vendor -o build/bind-api
    

FROM scratch as prod|\label{line:2nd-stage}|

ENV GIN_MODE=release

COPY --from=base /go/src/build/bind-api /usr/bin/bind-api

VOLUME [ "/data/bind/conf", "/data/bind/lib", "/data/api" ]

EXPOSE 2020

CMD ["bind-api"]
\end{lstlisting}

En una primera etapa (\verb|base|) se usa como capa la imagen la imagen de Go, \verb|golang:1.19.2|, a ella se copian las dependencias del proyecto previamente creadas por \verb|go mod vendor|. Una vez hecho esto, es compilado el sistema en un único binario con todas sus dependencias autocontenidas.

La segunda etapa (línea \ref{line:2nd-stage}) tiene como objetivo crear una imagen con el número mínimo de artefactos. Esto se logra copiando tan solo el binario compilado del código fuente en la primera etapa. A continuación, con vistas a la puesta en producción, son especificados los volúmenes y el puerto a usar. En esta se usa como base la imagen \verb|scratch|\footnote{\url{https://hub.docker.com/_/scratch}}, una imagen especial de Docker, la cual no crea una capa extra al ser usada como punto de partida y que no contiene archivos.

Con el empleo de esta aproximación se logra disminuir considerablemente el tamaño de la imagen resultante. La primera etapa da como resultado una imagen de 1.15GB de tamaño, mientras que la segunda tan solo 19MB. Se reduce el tamaño final aproximadamente un 98.3\% al eliminar el SDK de Go y otras herramientas que trae la imagen por defecto. Estas no son necesarias en producción, dado que la aplicación es compilada en un solo archivo binario que se comunica directamente con el sistema operativo.

\section{Experimentación}

Una vez concluida la propuesta y el desarrollo de su implementación, corresponde validar la solución creada. Para ello se modelan diferentes experimentos que verifiquen las funciones del servidor y su capacidad de respuesta. Con tal objetivo, se puede confirmar que la propuesta responde al problema para el que fue diseñada, pero además es posible verificar la carga a soportar de la implementación sobre BIND 9 y diferentes métricas que diagnostiquen la capacidad del servidor.

Primero, se trata de probar que añadir una nueva zona a través de la API, en efecto, modifica la configuración del servidor de nombres. Este experimento, además sirve para comprobar la latencia de la API al aumentar la cantidad de zonas que mantiene el servidor de BIND.

Segundo, se desea valorar el consumo de memoria por la API cuando el servidor maneja un número alto de zonas. Esto permite estimar la cantidad de zonas posibles a manejar de acuerdo a los recursos de cómputo.

Como tercera prueba, se realizan consultas DNS a BIND 9, de forma programática y manual para verificar que responde con la información requerida. Finalmente se muestran vistas de la interfaz web diseñadas que reflejan el uso de la API por esta para modificar la configuración de BIND 9.

Para la realización de los experimentos se usó la imagen de Docker usada en desarrollo y almacenada en el repositorio\footnote{\url{https://github.com/svex99/bind-api/blob/main/docks/dev.Dockerfile}}. Esta fue ejecutada en un ambiente con Ubuntu 22.04.1 LTS, procesador AMD Ryzen 5 5600h × 12, 8GB de RAM y 100GB de almacenamiento SSD.

\subsection{Latencia ante el aumento de la cantidad de zonas}

Para desarrollar este experimento el servidor fue poblado con una cantidad específica de zonas de forma previa. Cada fichero de zona creado tuvo un tamaño aproximado de 182B, lo que se traduce en una entrada en la configuración de BIND de aproximadamente 89B, y un total de 271B a procesar por cada zona. Cada una de las zonas solo tiene definidos tres registros, el \verb|SOA|, junto a un \verb|NS| y \verb|A| para indicar el servidor de nombres de la zonas. Esta información trae consigo un aumento de la carga en el servidor de BIND así como en la API, principalmente a la hora de ejecutar \verb|rndc reconfig| contra el servidor de nombres.

\begin{table}[!ht]
    \centering
    \begin{tabular}{|c|c|c|}
    \hline
    \multicolumn{1}{|p{4cm}|}{\centering\textbf{Zonas previas}} & \multicolumn{1}{|p{4cm}|}{\centering \textbf{Añadir 100 nuevas zonas (s)}} & \multicolumn{1}{|p{4cm}|}{\centering \textbf{Tiempo estimado por solicitud (ms)}} \\ \hline
        0 & 6.99 & 69.89 \\ \hline
        250 & 11.04 & 110.37 \\ \hline
        500 & 12.38 & 123.80 \\ \hline
        1000 & 14.15 & 141.51 \\ \hline
        2000 & 18.10 & 180.96 \\ \hline
        4000 & 24.66 & 246.55 \\ \hline
    \end{tabular}
    \caption{Tiempo de procesamiento de añadir 100 nuevas zonas de acuerdo al número de zonas previas.}
    \label{table:add-zones}
\end{table}

El tiempo necesario para dar respuesta ante un pedido es aceptable (tabla \ref{table:add-zones}), teniendo en cuenta la necesidad de comunicarse con el servidor de nombres. Este es el principal factor que aumenta el tiempo de respuesta, pues BIND 9 tiene que cargar cada una de las zonas en su configuración. El tiempo de respuesta aumenta a medida que toma más tiempo para el servidor procesar la solicitud de \verb|rdnc reconfig|. Además, la escritura en disco de la configuración cuando es añadida una nueva zona también aumenta, pues en sucesivas escrituras esta es reescrita.

La latencia de retribuir la lista completa de zonas, es baja (tabla \ref{table:get-zones}). La principal ventaja es que no se requiere de lectura en disco. La API mantiene una copia en memoria de la configuración del servidor de BIND con la cual responder a estos pedidos.

\begin{table}[!ht]
    \centering
    \begin{tabular}{|c|c|c|}
    \hline
    \multicolumn{1}{|p{4cm}|}{\centering\textbf{Zonas previas}} & \multicolumn{1}{|p{7cm}|}{\centering \textbf{Obtener la lista de zonas (ms)}} \\ \hline
        250  & 0.008740  \\ \hline
        500  & 0.019530  \\ \hline
        1000 & 0.056963  \\ \hline
        2000 & 0.174529  \\ \hline
        4000 & 0.182631  \\ \hline
    \end{tabular}
    \caption{Tiempo de procesamiento de obtener la lista de zonas de acuerdo al número de zonas previas.}
    \label{table:get-zones}
\end{table}

\subsection{Consumo de memoria ante el aumento de la cantidad de zonas}

Para medir el uso de memoria al cargar la información en memoria (invocar \verb|load|), se hizo uso de la biblioteca \verb|runtime/pprof|\footnote{\url{https://pkg.go.dev/runtime/pprof}} que viene por defecto en la instalación de Go. La herramienta del mismo nombre es usada para mostrar y analizar la información recogida en tiempo de ejecución. En este experimento se crearon zonas de forma semejante al anterior para desarrollar los pruebas.

\begin{figure}
    \centering
    \begin{subfigure}{0.49\textwidth}
        \includegraphics[width=\textwidth]{Graphics/mem250z.png}
        \caption{250 zonas}
    \end{subfigure}
    \hfill
    \begin{subfigure}{0.49\textwidth}
        \includegraphics[width=\textwidth]{Graphics/mem500z.png}
        \caption{500 zonas}
    \end{subfigure}
    \hfill
    \begin{subfigure}{0.49\textwidth}
        \includegraphics[width=\textwidth]{Graphics/mem1000z.png}
        \caption{1000 zonas}
    \end{subfigure}
    \hfill
    \begin{subfigure}{0.49\textwidth}
        \includegraphics[width=\textwidth]{Graphics/mem2000z.png}
        \caption{2000 zonas}
    \end{subfigure}
    \hfill
    \begin{subfigure}{0.49\textwidth}
        \includegraphics[width=\textwidth]{Graphics/mem4000z.png}
        \caption{4000 zonas}
    \end{subfigure}
    \hfill
    \begin{subfigure}{0.49\textwidth}
        \includegraphics[width=\textwidth]{Graphics/mem8000z.png}
        \caption{8000 zonas}
    \end{subfigure}

    \caption{Consumo de memoria de la implementación de \textbf{load} de acuerdo a la cantidad de zonas.}
    \label{fig:bench-mem}
\end{figure}

Se puede apreciar en la figura \ref{fig:bench-mem} que la cantidad de zonas que maneje el servidor de nombres incurre directamente en el consumo de memoria de la API. El consumo inicial con 250 zonas es de apenas unos MB, llegando a consumir casi 1GB de RAM cuando maneja 8000 zonas, aproximadamente el 94\% de la memoria usada por el proceso.

El consumo de memoria también es directamente proporcional a la información almacenada en cada zona. Sin embargo, algunos proveedores recomiendan mantener la cantidad de zonas en un servidor DNS por debajo de 10000 [\cite{dns-cap}]. Por lo que ante un consumo de memoria elevado, la solución potencial puede ser escalar de forma horizontal con otro servicio, aunque también es necesario tener en cuenta la carga sobre el servidor.

\subsection{Pruebas End-to-End}

Estas pruebas está diseñadas con el objetivo de verificar que el servidor de BIND 9 manejado desde la API sirve la información de forma correcta. La configuración para desarrollar dicha prueba esta compuesta por BIND 9 y la API, ubicados en una red privada de Docker, con el primero sirviendo consultas DNS externas a través de un puerto expuesto para conexiones TCP y UDP. El contenedor de la API usa como servidor DNS a BIND 9 con el objetivo de realizar las consultas de prueba contra él.

En este ambiente se implementa un grupo de pruebas que, de forma programática, invocan los puntos finales de la API y consultan al servidor DNS para comprobar el correcto funcionamiento. Al ejecutar el conjunto de pruebas en el ambiente interno de Docker, las consultas DNS son resueltas de forma correcta con el servidor DNS. 

\begin{lstlisting}[frame=single, numbers=none, caption=Resultados del conjunto de pruebas \textit{end-to-end}.]
--- PASS: TestAPIFlow (8.73s)
    --- PASS: TestAPIFlow/TestAddZone (1.43s)
    --- PASS: TestAPIFlow/TestMXRecord (0.42s)
        --- PASS: TestAPIFlow/TestMXRecord/TestAddMX (0.34s)
        --- PASS: TestAPIFlow/TestMXRecord/TestDelMX (0.08s)
    --- PASS: TestAPIFlow/TestCNAMERecord (0.33s)
        --- PASS: TestAPIFlow/TestCNAMERecord/TestAddCNAME (0.26s)
        --- PASS: TestAPIFlow/TestCNAMERecord/TestDelCNAME (0.08s)
    --- PASS: TestAPIFlow/TestTXTRecord (0.43s)
        --- PASS: TestAPIFlow/TestTXTRecord/TestAddTXT (0.13s)
        --- PASS: TestAPIFlow/TestTXTRecord/TestDelTXT (0.30s)
    --- PASS: TestAPIFlow/TestDelZone (6.09s)
PASS
ok      github.com/svex99/bind-api      8.745s
\end{lstlisting}

Desde la computadora que realiza las pruebas y externo al ambiente de Docker, también es verificada esta información usando la herramienta \verb|dig|\footnote{\url{https://man.openbsd.org/dig.1}} para realizar consultas al servidor de nombres. A continuación, se toma como referencia un archivo de zona, y las consultas realizadas al servidor relacionadas con él.

\begin{lstlisting}[frame=single, numbers=none, caption=Archivo de registros para la zona \textbf{flow-test-1.com} servida por el servidor de BIND 9.]
$ORIGIN flow-test-1.com.
$TTL 2d
@ IN SOA ns1 svex ( 2022112514 86400 7200 3600000 172800 )
@ IN NS ns1
ns1 IN A 10.1.3.1
@ IN MX 101 email-server1
@ IN MX 103 email-server3
cname-dest IN A 7.7.7.7
@ IN TXT "key1=value-body"
\end{lstlisting}

\begin{lstlisting}[frame=single, numbers=none, caption=Consulta externas realizadas al servidor para la zona \textbf{flow-test-1.com}.]
$ dig +short @localhost -p 30053 flow-test-1.com soa
ns1.flow-test-1.com. svex.flow-test-1.com. 2022112514 86400 7200 3600000 172800

$ dig +short @localhost -p 30053 ns1.flow-test-1.com
10.1.3.1

$ dig +short @localhost -p 30053 flow-test-1.com mx 
101 email-server1.flow-test-1.com.
103 email-server3.flow-test-1.com.

$ dig +short @localhost -p 30053 flow-test-1.com txt 
"key1=value-body"
\end{lstlisting}

La capacidad del servidor DNS para responder a cada una de las consultas realizadas, tanto programáticas como manuales, valida el funcionamiento del flujo diseñado. Siendo así efectiva la solución para administrar el servidor de nombres de BIND 9.

\subsection{Funcionalidad de la Web}

La interfaz web hace uso de los diferentes puntos finales de la API para modificar el servidor de BIND 9. El CRUD de los diferentes objetos del servidor de nombres puede realizarse a través de esta interfaz sin necesidad de interactuar con la API directamente (Figura \ref{fig:web-view}).

\begin{figure}
    \centering
    \begin{subfigure}{0.49\textwidth}
        \includegraphics[width=\textwidth]{Graphics/web/list-zones.png}
        \caption{Vista para listar las diferentes zonas.}
    \end{subfigure}
    \hfill
    \begin{subfigure}{0.49\textwidth}
        \includegraphics[width=\textwidth]{Graphics/web/add-zone.png}
        \caption{Formulario para añadir una nueva zona.}
    \end{subfigure}
    \hfill
    \begin{subfigure}{0.49\textwidth}
        \includegraphics[width=\textwidth]{Graphics/web/list-records.png}
        \caption{Vista para administrar los detalles de una zona y los diferentes registros.}
    \end{subfigure}
    \hfill
    \begin{subfigure}{0.49\textwidth}
        \includegraphics[width=\textwidth]{Graphics/web/edit-records.png}
        \caption{Opción para editar los registros.}
    \end{subfigure}

    \caption{Diferentes vistas de las página web.}
    \label{fig:web-view}
\end{figure}


\subsection{Discusión}

Los experimentos dan como resultado que la implementación de la propuesta es válida para su puesta en producción. Por tanto, un aproximación semejante en base a la propuesta puede aplicarse a otros servidores DNS autoritarios de código abierto. A pesar de esto, se considera que cada solución debe adaptarse a las cualidades del software al que aplica, como fue este caso con BIND 9.

En el entorno de la Universidad de La Habana, y organizaciones en general, las modificaciones ejecutadas sobre un servicio DNS a diario no llegan a ser altas para la capacidad de respuesta de la solución. Dado este caso, es válido aceptar como buenos los valores de latencia al hacer modificaciones sobre BIND y, por tanto, no es necesario considerar una optimización prematura sobre la misma.

El consumo de memoria es el principal aspecto a considerar a la hora de la puesta en producción, pero solo es acorde a la configuración que maneja el servidor de BIND. Positivamente, ofrece como ventaja obtener un tiempo de respuesta más rápido por parte de la API ante los pedidos.

La posibilidad de haber preparado la API para su despliegue con Docker permite limitar los recursos que esta usa a nivel de sistema. Por tanto, en el caso que la API y BIND 9 compartan recursos en una misma computadora y no sea posible estimarse el uso de memoria por la API, es posible desplegar el primero, limitando su uso de memoria, sin interferir con el funcionamiento del servicio DNS.

La interfaz web ofrece la facilidad de interactuar con la configuración del servidor de nombres y manejar tanto las zonas como el conjunto de registros que estas poseen. Cada una de las funciones derivadas de la función \verb|save| propuesta es usada indirectamente para responder a las modificaciones realizadas desde la interfaz web. Al ofrecer una aplicación de página simple la navegación es fluida una vez cargada la página inicial. Adicionalmente, el diseño es responsivo lo que permite la configuración desde un teléfono móvil o pantallas pequeñas.